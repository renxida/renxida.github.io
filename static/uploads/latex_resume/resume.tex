\documentclass[10pt,letterpaper]{article}
\usepackage[margin=0.2in]{geometry}
\usepackage{enumitem}
\usepackage{hyperref}
\usepackage{titlesec}
\usepackage{bold-extra}
\usepackage[T1]{fontenc}

% Remove page numbers
\pagestyle{empty}

% Section formatting
\titleformat{\section}{\normalsize\bfseries\scshape}{}{0em}{}[\titlerule]
\titlespacing{\section}{0pt}{2pt}{2pt}

% Subsection formatting
\titleformat{\subsection}[runin]{\bfseries}{}{0em}{}
\titlespacing{\subsection}{0pt}{0pt}{0.5em}

% Remove paragraph indentation
\setlength{\parindent}{0pt}

% Reduce list spacing
\setlist[itemize]{leftmargin=*,noitemsep,topsep=0pt,parsep=0pt,partopsep=0pt,itemsep=-2pt}

\begin{document}

% Header
\begin{center}
    {\Large \textbf{Cedar Ren}} \\
    (757) 279-4582 \hspace{0.3cm} | \hspace{0.3cm} \href{https://renresear.ch}{renresear.ch} \hspace{0.3cm} | \hspace{0.3cm} \href{mailto:cedar.ren@gmail.com}{cedar.ren@gmail.com}
\end{center}

\section{Education}

\textbf{University of Virginia} \hfill Sep 2019 -- Jan 2023 \\
\textit{Dropped out of PhD in Computer Science} | Computer Architecture \& some ML | GRE: Verbal:167, Quant:168 \hfill \textit{GPA}: 4.0

\textbf{College of William and Mary} \hfill August 2016 -- May 2019 \\
\textit{Bachelors of Science} | Double Major in Computer Science and Mathematics \hfill \textit{GPA}: 3.8

\section{Skills}

\textbf{Programming Languages:} Rust, C++, Python, C, Zig, SQL, MatLab, Lean \\
\textbf{Software:} Torch, Numpy, Keras, TensorFlow, sklearn, Linux daily driver \\
\textbf{Specializations:} Machine Learning, Queuing Theory, Statistics, Probability, CPU/GPU Performance Profiling \\
\textbf{Interests:} Security, Hardware Accelerators, Formal Verification, Ursula K LeGuin books, FengShui

\section{Work Experience}

\vspace{3pt}
\textbf{MatX} -- Software Stack for Transformer ASICs \hfill Apr 2025 -- Current
\begin{itemize}
    \item Provide feedback for hardware design by developing cycle-accurate simulator and modeling performance requirements of recent LLMs (Deepseek, Kimi, GPT-OSS)
    \item Develop research kernels and compilers to support hardware design.
\end{itemize}

\vspace{3pt}
\textbf{AMD} -- Open source ML compiler stack \hfill Sep 2023 -- Apr 2025
\begin{itemize}
    \item Lead bring-up of performant serving stack for pre-compiled LLMs by writing the CI setup, implementing prefix-sharing kv caching, and squashing numerous bugs. See list of PRs \href{https://github.com/nod-ai/shark-ai/pulls?q=is\%3Apr+author\%3Arenxida}{here}.
    \item Enabled performant compilation of LLMs on Vulcan and ROCm via contributions to \href{https://github.com/llvm/torch-mlir/pulls?q=is\%3Apr+author\%3Arenxida}{torch-mlir}, \href{https://github.com/iree-org/iree}{IREE} \& \href{https://github.com/nod-ai/SHARK-TestSuite/pulls?q=is\%3Apr+author\%3Arenxida}{SHARK-TestSuite}
\end{itemize}

\vspace{3pt}
\textbf{Stealth Mode Crypto Data Company / Consulting role} \hfill Mar-Sep 2023
\begin{itemize}
    \item Improved Zig Etherium implementation performance by $\sim$2x by migrating to an 100x faster AVX512 vectorized hash function
    \item Detect performance hotspots w/ Tracy with performance counters + instrumentation
\end{itemize}

\vspace{3pt}
\textbf{Intel Labs} -- Architecture Tooling Group | Research Intern \hfill Oct 2022 -- Jan 2023
\begin{itemize}
    \item Accelerate SimPoint generation by 200x using hardware performance counters sampling to avoid instrumentation
    \item Achieve $<$3\% CPI and $<$10\% MPKI estimation accuracy while retaining 1,000,000x benchmarking speedup from SimPoint
\end{itemize}

\vspace{3pt}
\textbf{University of Virginia} -- Computer Science Department | PhD Candidate \hfill Sep 2019 -- Jan 2023
\begin{itemize}
    \item Applied formal verification to ensure that quantized machine learning models remained invulnerable to adversarial attacks using DNNV (\href{https://github.com/dlshriver/dnnv}{https://github.com/dlshriver/dnnv}), ONNX, and ReluPlex (\href{https://arxiv.org/abs/1702.01135}{https://arxiv.org/abs/1702.01135})
    \item Discovered 2 critical security flaws that threatened execution integrity and data security in modern x86 processors.
    \item Mentored 5 undergraduate students on computer architecture and machine learning projects, breaking down large projects into digestible chunks, as well as providing instruction on computer architecture, side-channel attacks, machine learning compilers, and ML models (incl. model specification, feature engineering, parameter tuning, and cross-validation).
\end{itemize}

\vspace{3pt}
\textbf{NXP Semiconductors} -- Edge Security | ML Research Intern \hfill May 2022 -- Aug 2022
\begin{itemize}
    \item Applied statistical and machine learning algorithms (\textit{incl. logistic regression, perceptrons, time-convolutional neural networks, decision trees, k-nearest neighbors, random forests, support vector regressions}) to monitor CPU performance counters for Spectre and Meltdown type side-channel attacks (\textit{Python, scikit-learn, pandas, statsmodels, NumPy, MLJar})
\end{itemize}

\section{Selected Projects}

\vspace{3pt}
\textbf{ProxyVM} -- In collaboration with Intel Labs and the Semiconductor Research Corporation \hfill Jan 2022 - Jan 2023
\begin{itemize}
    \item Augment profiling tools with differential privacy to enable ML hardware supply chain collaboration without loss of privacy
    \item Accelerated pre-silicon hardware simulations while maintaining high performance predictability by generating augmented performance traces (basic block vectors augmented with data access pattern vectors).
    \item Extended existing system to emerging hardware and workloads using LLVM and MLIR as a compatibility layer
    \item Modify cross-platform machine learning compiler to generate execution traces for benchmarking on CPU, GPU, FPGA, and ASIC
\end{itemize}

\vspace{3pt}
\textbf{I See Dead Micro-Ops} \hfill Sep 2019 -- Jan 2021
\begin{itemize}
    \item Analyzed Intel x86 processor design documents to discover potential vulnerability, craft microbenchmarks to reverse undocumented CPU features, and design proof-of-concept exploits for novel vulnerabilities
    \item Designed micro-architectural benchmarks that characterized undocumented x86 instruction translation mechanisms
    \item Published novel spectre-type attack in \href{https://ieeexplore.ieee.org/document/9499837}{International Symposium on Computer Architecture} (15\% acceptance rate).
    \item Published SMT performance-preserving speculative side-channel defenses to protect processors without compromising performance. \href{https://www.usenix.org/conference/usenixsecurity22/presentation/taram}{USENIX security} (18\% acceptance rate)
\end{itemize}

\vspace{3pt}
\textbf{Equity AI} \textit{(Honors Thesis, Summa Cum Laude)} \hfill Dec 2017 -- May 2019
\begin{itemize}
    \item Distributed bayesian hyperparameter optimization over a small ($\sim$50 machines) cluster, using idle lab computing.
\end{itemize}

\vspace{3pt}
\textbf{Int'l Genetically Engineered Machines Contest} \textit{(iGEM 2017, Int'l 2nd Place \& Best Model Award)} \hfill Oct 2016 -- Sep 2017
\begin{itemize}
    \item Won 2nd place overall \& best math model in \href{https://2017.igem.org/Team:William_and_Mary/Speed_Control}{international genetic engineering contest} (iGEM 2017)
    \item Modeled genetic circuits with partial differential models + MCMC, testing predictions against wet-lab experimental results
    \item Designed plasmid to implement protein-protease gene circuit to demonstrate novel gene expression rate control method
\end{itemize}

\end{document}